\documentclass[paper=a4, fontsize=14pt]{scrartcl} % A4 paper and 11pt font size

\newcommand{\horrule}[1]{\rule{\linewidth}{#1}} % Create horizontal rule command with 1 argument of height

\title{	
\huge The Sudoku Puzzle \\ % The assignment title
\horrule{0.5pt} \\[0.5cm] % Thick bottom horizontal rule
}

\author{Thomas Del Prete, \\
		Alessandra H{\" a}dener, \\
		Simone Masiero, \\
		Yanna Poncioni, \\
		Damiano Pugliesi} % Your name

\date{\normalsize\today} % Today's date or a custom date

\begin{document}
	
	\maketitle
	
	\section*{Introduction}
	
		\textbf{Sudoku} is a placement puzzle, also known as Number Place in the U.S.A.
		\newline
		The game consists most frequently of a 9 x 9 grid, divided in 9 subgrids with dimension 3 x 3 called ``regions".
		\newline
		The purpose is to enter a digit from 1 to 9 (or other symbols e.g. letters, icons) in each cell of the grid so that each row, column and region contains only one instance of each digit.
		\newline
		
		
	\section*{Reduces Sudoku problem to a SAT clause}
	
	\section*{SAT Solver}
	
	\section*{Conclusion}
	
	
\end{document}


