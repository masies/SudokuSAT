\documentclass[paper=a4, fontsize=14pt]{scrartcl} % A4 paper and 11pt font size

\newcommand{\horrule}[1]{\rule{\linewidth}{#1}} % Create horizontal rule command with 1 argument of height

\title{	
\huge The Sudoku Puzzle \\ % The assignment title
\horrule{0.5pt} \\[0.5cm] % Thick bottom horizontal rule
}

\author{Thomas Del Prete, \\
		Alessandra H{\" a}dener, \\
		Simone Masiero, \\
		Yanna Poncioni, \\
		Damiano Pugliesi} % Your name

\date{\normalsize\today} % Today's date or a custom date

\begin{document}
	
	\maketitle
	
	\section*{Introduction}
	
		\textbf{Sudoku} is a placement puzzle, also known as Number Place in the U.S.A.
		\newline
		The game consists most frequently of a 9 x 9 grid, divided in 9 subgrids with dimension 3 x 3 called ``regions".
		\newline
		The purpose is to enter a digit from 1 to 9 (or other symbols e.g. letters, icons) in each cell of the grid so that each row, column and region contains only one instance of each digit.
		\newline
		We implemented a \textbf{SAT solver} (Instead of combine backtracking and methods for constraint propagation as other Sudoku solver) to figured out a correct solution for the Sudoku.
		\newline
		Basically the Sudoku is translated into a propositional formula that can be satisfy only if the Sudoku has a solution.
		\newline
		Once the propositional formula is formulated, The SAT solver tries to find a satisfying assignment that will become the solution for the original Sudoku.
			 			
	\section*{Reduces Sudoku problem to a SAT clause}
	
	
	Digits are modelled by a datatype with nine elements $(1, . . , 9)$.
	\newline
	We can say that the grid cells $(x1, . . , x9)$ are valid if they contain
	every digit.
	\subsubsection*{Definition 1}
		\[ \operatorname{valid}\left(x_{1}, x_{2}, x_{3}, x_{4}, x_{5}, x_{6}, x_{7}, x_{8}, x_{9}\right) \equiv \bigwedge_{d=1}^{9} \bigvee_{i=1}^{9} x_{i}=d \]
	Labeling the 81 cells, we can check if the cells generate a correct solution for the Sudoku puzzle.
	
	\subsubsection*{Sudoku definition}
	\[\operatorname{sudoku}\left(\left\{x_{i j}\right\}_{i, j \in\{1, \ldots, 9\}}\right) \equiv \bigwedge_{i=1}^{9} \operatorname{valid}\left(x_{i 1}, x_{i 2}, x_{i 3}, x_{i 4}, x_{i 5}, x_{i 6}, x_{i 7}, x_{i 8}, x_{i 9}\right)\]
	
	\[ \wedge \bigwedge_{j=1}^{9} \operatorname{valid}\left(x_{1 j}, x_{2 j}, x_{3 j}, x_{4 j}, x_{5 j}, x_{6 j}, x_{7 j}, x_{8 j}, x_{9 j}\right)\]
	
	\[ \wedge \bigwedge_{i, j \in\{1,4\}} \operatorname{valid}\left(x_{i j}, x_{i(j+1)}, x_{i(j+2)}, x_{(i+1) j}, x_{(i+1)(j+1)}, x_{(i+1)(j+2),}\right.\]
	
	\[ x_{(i+2) j}, x_{(i+2)(j+1)}, x_{(i+2)(j+2)} )\]
	
	\section*{SAT Solver}
	We're now introducing 9 boolean variables for each cell of the 9x9 grid in order to encode a Sudoku.
	\newline
	Each boolean value holds the truth value of the equation x$_{i,j}$ = d.
	\newline A clause \[\bigvee_{d=1}^{9} p_{i j}^{d}\] assures that a cell contains one of the nine accepted digits, whilst 36 clauses
	\newline
	\[\bigwedge_{1 \leq d<d^{\prime} \leq 9} \neg p_{i j}^{d} \vee \neg p_{i j}^{d^{\prime}} \]
	assure that a cell doesn't hold two different digits.
	\newline
	Since the number of digits is equals to the number of cells in every row, column or region, then the nine grid cells (x$_1$, ..., x$_9$) hold distinct values.
	\newline
	\subsubsection*{Lemma 1} \\
	\[\operatorname{valid}\left(x_{1}, x_{2}, x_{3}, x_{4}, x_{5}, x_{6}, x_{7}, x_{8}, x_{9}\right) \Longleftrightarrow \bigwedge_{1 \leq i<j \leq 9} x_{i} \neq x_{j}\]
	\[\Longleftrightarrow \bigwedge_{1 \leq i<j \leq 9} \bigwedge_{d=1}^{9} x_{i} \neq d \vee x_{j} \neq d\]
	\newline
The given formula, when converted into \texttt{SAT}, is translated into 9 clauses * 36 inequations = 324 clauses, each of length 2. This allows more unit propagation at the boolean level than what was previously stated in \textbf{Def. 1}, which gives us the possibility of cross hatching digits (a technique used in Sudoku to reduce the search space).
\newline
Summarising, up to now, our \texttt{SAT} is composed of:
	\begin{enumerate}
		\item 81 definedness clauses of length 9
		\item 81 $\cdot$ 36 uniqueness clauses of length 2
		\item 27 $\cdot$ 324 validity clauses of length 2
	\end{enumerate}
For a total of \textbf{11745} clauses.
\newline
It has to be noticed, though, the we don't need boolean values for cells who's value is already contained in the Sudoku to be solved. Following this, definedness, uniqueness and some validity clauses can be omitted from these cells. Hence the actual number of variables and clauses will be less than 729 and 11745 respectively.
\newline
Finally, our encoding produced a propositional formula already in \texttt{CNF}, so the conversion into \texttt{DIMACS CNF} (the input format for most sat solvers) is trivial.
\end{document}


